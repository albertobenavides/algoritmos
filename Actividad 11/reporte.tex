% https://es.overleaf.com/latex/templates/project-report/jpzczmpsdzwm

%%% Preamble
\documentclass[paper=leter, fontsize=11pt]{scrartcl}
\usepackage[utf8]{inputenc}
\usepackage[spanish,mexico]{babel}
\usepackage[T1]{fontenc}    % use 8-bit T1 fonts
\usepackage{lmodern}
\usepackage{hyperref}       % hyperlinks
\usepackage{lipsum}
\usepackage[square,numbers]{natbib}

\usepackage[protrusion=true,expansion=true]{microtype}	
\usepackage{amsmath,amsfonts,amsthm} % Math packages
\usepackage[pdftex]{graphicx}
\usepackage{url}

\usepackage{booktabs}

\usepackage{tikz}

\usepackage{caption}
\usepackage{subcaption}

\usepackage{listings}
\lstdefinestyle{mystyle}{
    basicstyle=\ttfamily\footnotesize,
    breakatwhitespace=false,         
    breaklines=true,                 
    captionpos=b,                    
    keepspaces=true,                 
    numbers=left,                    
    numbersep=5pt,                  
    showspaces=false,                
    showstringspaces=false,
    showtabs=false,                  
    tabsize=4
}

\lstset{style=mystyle}
\renewcommand{\lstlistingname}{Código}

\graphicspath{ {img/} }

\selectlanguage{spanish}
\usepackage[spanish,onelanguage,ruled]{algorithm2e}


%%% Custom sectioning
\usepackage{sectsty}
\allsectionsfont{\centering \normalfont\scshape}


%%% Custom headers/footers (fancyhdr package)
\usepackage{fancyhdr}
\pagestyle{fancyplain}
\fancyhead{}											% No page header
\fancyfoot[L]{}											% Empty 
\fancyfoot[C]{}											% Empty
\fancyfoot[R]{\thepage}									% Pagenumbering
\renewcommand{\headrulewidth}{0pt}			% Remove header underlines
\renewcommand{\footrulewidth}{0pt}				% Remove footer underlines
\setlength{\headheight}{13.6pt}


%%% Equation and float numbering
\numberwithin{equation}{section}		% Equationnumbering: section.eq#
\numberwithin{figure}{section}			% Figurenumbering: section.fig#
\numberwithin{table}{section}				% Tablenumbering: section.tab#


%%% Maketitle metadata
\newcommand{\horrule}[1]{\rule{\linewidth}{#1}} 	% Horizontal rule

%%% https://tex.stackexchange.com/a/118217
\usepackage{mathtools}
\DeclarePairedDelimiter\ceil{\lceil}{\rceil}
\DeclarePairedDelimiter\floor{\lfloor}{\rfloor}

\usepackage{amsmath}

\usepackage{tikz}

\title{
		%\vspace{-1in} 	
		\usefont{OT1}{bch}{b}{n}
		\normalfont \normalsize \textsc{Posgrado de Ingeniería de Sistemas} \\ [25pt]
		\horrule{0.5pt} \\[0.4cm]
		\huge Unir-encontrar \\
		\horrule{2pt} \\[0.5cm]
}
\author{
		\normalfont 								\normalsize
        Alberto Benavides\\[-3pt]		\normalsize
        \today
}
\date{}


%%% Begin document
\begin{document}
\maketitle

\section{Introducción}

Las operaciones de \textit{unir-encontrar} permiten realizar funciones de búsqueda y unión en conjuntos que tengan elementos con identificadores irrepetibles entre sí. Una manera de representar conjuntos y sus elementos es mediante grafos en que cada nodo corresponde a cada elemento, mientras que grafos disjuntos equivalen a conjuntos, lo cual aparece representado en los incisos de la figura \ref{tres_grafos}.


\begin{figure}
    \begin{subfigure}{.45\textwidth}
        \centering
        \begin{tikzpicture}
            \node[draw, circle] at (0, 0) {$e$};
        \end{tikzpicture}
        \caption{Un conjunto $C = \{e\}$ representado por un nodo}
        \label{nodo}
    \end{subfigure}
    \hfill
    \begin{subfigure}{.45\textwidth}
        \centering
        \begin{tikzpicture}
            \node[draw, circle] (e1) at (0, 0) {$e_1$};
            \node[draw, circle] (e2) at (2, 0) {$e_2$};
            \node[draw, circle] (e3) at (4, 0) {$e_3$};
            
            \draw[] (e1) -- (e2);
            \draw[] (e2) -- (e3);
        \end{tikzpicture}
        \caption{Un conjunto \(C = \{e_1, e_2, e_3\}\) representado por un grafo con tres nodos.}
        \label{grafo}
    \end{subfigure}

    \vspace{1cm}
    
    \centering
    \begin{subfigure}{0.45\textwidth}
        \centering
        \begin{tikzpicture}
            \node[draw, circle] (e1) at (0, 0) {$e_1$};
            \node[draw, circle] (e2) at (2, 0) {$e_2$};
            \node[draw, circle] (e3) at (4, 0) {$e_3$};
            
            \draw[] (e2) -- (e3);
        \end{tikzpicture}
        \caption{Los conjuntos \(C_1 = \{e_1\}\) y \(C_2 = \{e_2, e_3\}\) representados por dos grafo.}
        \label{grafos}
    \end{subfigure}

    \caption{Ejemplos de representación de conjuntos y elementos por grafos.}
    \label{tres_grafos}
\end{figure}

En este caso, la unión de dos conjuntos se realiza mediante la función de búsqueda primeramente. Esta última función consiste en devolver todos los elementos pertenecientes a un conjunto. Una manera de implementar esto es construir un diccionario en el que a cada elemento se le asigne un índice $i = \{0, 1, 2, \ldots\}$ y se asocie con el índice $p$ un nodo de otro grafo al que se haya unido. A este otro nodo se le denomina \textit{padre}, por comodidad. Cuando un elemento se tiene por padre a sí mismo se considera que forma parte de un conjunto con un solo elemento. Esto se ejemplifica en la tabla \ref{asignacion}.

\begin{table}[]
    \centering
    \caption{Ejemplo de asignación de índices a elementos de un conjunto.}
    \label{asignacion}
    \begin{tabular}{@{}ccccc@{}}
    \toprule
    Elemento & $e_1$ & $e_1$ & $e_1$ & $\ldots$ \\ \midrule
    Padre   & $0$     & $1$     & $2$     & $\ldots$ \\ \bottomrule
    Índice   & $0$     & $1$     & $2$     & $\ldots$ \\ \bottomrule
    \end{tabular}
\end{table}

Unir dos conjuntos a partir de los elementos $e_1 \in C_1$ y $e_2 \in C_2$ consiste en definir qué nodo será padre del otro y luego recorrer todos los padres del nodo hijo hasta encontrar uno que no tenga ningún padre, lo que puede hacerse en tiempo $\mathcal{O}(n)$

Estas implementaciones pueden consultarse en \url{https://jbenavidesv87.github.io/algoritmos/union-find.html}.

\bibliographystyle{plainnat}
\bibliography{Biblio}

\end{document}
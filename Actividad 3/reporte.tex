% https://es.overleaf.com/latex/templates/project-report/jpzczmpsdzwm

%%% Preamble
\documentclass[paper=leter, fontsize=11pt]{scrartcl}
\usepackage[utf8]{inputenc}
\usepackage[spanish,mexico]{babel}
\usepackage[T1]{fontenc}    % use 8-bit T1 fonts
\usepackage{lmodern}
\usepackage{hyperref}       % hyperlinks
\usepackage{lipsum}

\usepackage[protrusion=true,expansion=true]{microtype}	
\usepackage{amsmath,amsfonts,amsthm} % Math packages
\usepackage[pdftex]{graphicx}	
\usepackage{url}

\usepackage{caption}
\usepackage{subcaption}

\graphicspath{ {.} }

\selectlanguage{spanish}
\usepackage[spanish,onelanguage]{algorithm2e}


%%% Custom sectioning
\usepackage{sectsty}
\allsectionsfont{\centering \normalfont\scshape}


%%% Custom headers/footers (fancyhdr package)
\usepackage{fancyhdr}
\pagestyle{fancyplain}
\fancyhead{}											% No page header
\fancyfoot[L]{}											% Empty 
\fancyfoot[C]{}											% Empty
\fancyfoot[R]{\thepage}									% Pagenumbering
\renewcommand{\headrulewidth}{0pt}			% Remove header underlines
\renewcommand{\footrulewidth}{0pt}				% Remove footer underlines
\setlength{\headheight}{13.6pt}


%%% Equation and float numbering
\numberwithin{equation}{section}		% Equationnumbering: section.eq#
\numberwithin{figure}{section}			% Figurenumbering: section.fig#
\numberwithin{table}{section}				% Tablenumbering: section.tab#


%%% Maketitle metadata
\newcommand{\horrule}[1]{\rule{\linewidth}{#1}} 	% Horizontal rule

\title{
		%\vspace{-1in} 	
		\usefont{OT1}{bch}{b}{n}
		\normalfont \normalsize \textsc{Posgrado de Ingeniería de Sistemas} \\ [25pt]
		\horrule{0.5pt} \\[0.4cm]
		\huge Máquinas de Turing \\
		\horrule{2pt} \\[0.5cm]
}
\author{
		\normalfont 								\normalsize
        Alberto Benavides\\[-3pt]		\normalsize
        \today
}
\date{}


%%% Begin document
\begin{document}
\maketitle
\section{Máquinas de Turing}
Una máquina de Turing $M = (K, \Sigma, \delta, s)$ está definida por un conjunto de estados $K$ entre los que se incluyen ``alto'', ``sí'' y ``no''; un alfabeto $\Sigma$ que engloba un conjunto de símbolos entre los que deben figurar $\triangleright$ como símbolo inicial y $\sqcup$ como símbolo vacío; una función de transición $\delta(q, \sigma) = (p, \rho, D)$ en que $q$ es el estado actual, $\sigma$ símbolo bajo el puntero, $p$ el nuevo estado, $\rho$ el símbolo que sobreescribe a $\sigma$ y $D$ la dirección en que se moverá el puntero dadas la izquierda $\leftarrow$, derecha $\rightarrow$ y el no moverse $-$.

En este reporte se mostrarán algunos ejemplos de máquinas de Turing.

\section{Capicúa binario}
Esta máquina de Turing determina si un número binario dado es capicúa, o sea, que el número es el mismo leído de derecha a izquierda o de izquierda a derecha, por ejemplo son capicúas $1, 11, 101, 10101$ y no lo son $10, 110, 10100$. En este caso, el alfabeto $\Sigma = \{ 0, 1, \triangleright, \sqcup \}$ y el conjunto de estados $K = \{ a, b, c, d, e, f, \text{alto}, \text{sí}, \text{no} \}$. La función de transición para esta máquina se encuentra en la tabla \ref{capicua} (p. \pageref{capicua}).

\begin{table}[]
	\caption{Máquina de Turing para detectar capicúas binarios}
	\label{capicua}
	\centering
	\begin{tabular}{cc|c}
	$p$ & $\sigma$ & $\delta(p, \sigma)$ \\ \hline
	$a$ & $\triangleright$ & $(a, \triangleright, \rightarrow)$ \\
	$a$ & $0$ & $(b, \triangleright, \rightarrow)$ \\
	$a$ & $1$ & $(c, \triangleright, \rightarrow)$ \\
	$b$ & $\triangleright$ & $(\text{si}, \triangleright, -)$ \\
	$b$ & $0$ & $(b, 0, \rightarrow)$ \\
	$b$ & $1$ & $(b, 1, \rightarrow)$ \\
	$b$ & $\sqcup$ & $(d, \sqcup, \leftarrow)$ \\
	$c$ & $\triangleright$ & $(\text{si}, \triangleright, -)$ \\
	$c$ & $0$ & $(c, 0, \rightarrow)$ \\
	$c$ & $1$ & $(c, 1, \rightarrow)$ \\
	$c$ & $\sqcup$ & $(e, \sqcup, \leftarrow)$ \\
	$d$ & $\triangleright$ & $(\text{si}, \triangleright, -)$ \\
	$d$ & $0$ & $(f, \sqcup, \leftarrow)$ \\
	$d$ & $1$ & $(\text{no}, 1, -)$ \\
	$e$ & $\triangleright$ & $(\text{si}, \triangleright, -)$ \\
	$e$ & $0$ & $(\text{no}, 0, -)$ \\
	$e$ & $1$ & $(f, \sqcup, \leftarrow)$ \\
	$f$ & $0$ & $(f, 0, \leftarrow)$ \\
	$f$ & $1$ & $(f, 1, \leftarrow)$ \\
	$f$ & $\sqcup$ & $(a, \sqcup, \leftarrow)$
	\end{tabular}
\end{table}


\section{Duplicado de 1s}

Ahora se definirá una máquina de Turing que, dada una secuencia de $1$s, regrese una secuencia que contenga los $1$s iniciales, un espacio vacío y el doble de los $1$s iniciales. Es decir, si se parte de $11$, se debe retornar $11 \sqcup 1111$. Para esta máquina se tiene el alfabeto $\Sigma = \{ 0, 1, *, \triangleright, \sqcup \}$ y los estados $K = \{ a, b, c, d, e, f, \text{alto} \}$. Esta máquina cuenta con una función de transición desplegada en la tabla \ref{duplicar} (p. \pageref{duplicar}).

\begin{table}[]
	\caption{Máquina de Turing para duplicar $1$s}
	\label{duplicar}
	\centering
	\begin{tabular}{cc|c}
	$p$ & $\sigma$ & $\delta(p, \sigma)$ \\ \hline

	$a$ & $\triangleright$ & $(a, \triangleright, \rightarrow)$ \\
	$a$ & $1$ & $(b, 0, \rightarrow)$ \\
	$a$ & $\sqcup$ & $(f, \sqcup, \rightarrow)$ \\
	$b$ & $1$ & $(b, 1, \rightarrow)$ \\
	$b$ & $\sqcup$ & $(c, *, \rightarrow)$ \\
	$c$ & $0$ & $(c, 0, \rightarrow)$ \\
	$c$ & $\sqcup$ & $(d, 0, \rightarrow)$ \\
	$d$ & $0$ & $(d, 0, \leftarrow)$ \\
	$d$ & $*$ & $(e, \sqcup, \leftarrow)$ \\
	$d$ & $\sqcup$ & $(d, 0, \leftarrow)$ \\
	$e$ & $0$ & $(a, 1, \rightarrow)$ \\
	$e$ & $1$ & $(a, 1, \leftarrow)$ \\
	$f$ & $0$ & $(f, 1, \rightarrow)$ \\
	$f$ & $\sqcup$ & $(\text{alto}, \sqcup, -)$
	\end{tabular}
\end{table}

\section{Reflexiones finales}
Una máquina de Turing que determina palíndromos utiliza una secuencia de estados por cada letra que quiere ser incluida en el alfabeto (adicionales a las letras especiales). Esta es una manera de recordar o almacenar información. 

Otra manera de almacenar información es mediante letras de un alfabeto, como en la máquina de Turing que duplica $1$s, en que se utilizan $0$s para recordar los espacios que deben llenarse con $1$s en el estado $e$ y en el estado $f$.

\bibliographystyle{plainnat}
\bibliography{../Biblio}

\end{document}